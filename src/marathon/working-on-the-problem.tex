\PassOptionsToPackage{unicode=true}{hyperref} % options for packages loaded elsewhere
\PassOptionsToPackage{hyphens}{url}
%
\documentclass[]{article}
\usepackage{lmodern}
\usepackage{amssymb,amsmath}
\usepackage{ifxetex,ifluatex}
\usepackage{fixltx2e} % provides \textsubscript
\ifnum 0\ifxetex 1\fi\ifluatex 1\fi=0 % if pdftex
  \usepackage[T1]{fontenc}
  \usepackage[utf8]{inputenc}
  \usepackage{textcomp} % provides euro and other symbols
\else % if luatex or xelatex
  \usepackage{unicode-math}
  \defaultfontfeatures{Ligatures=TeX,Scale=MatchLowercase}
\fi
% use upquote if available, for straight quotes in verbatim environments
\IfFileExists{upquote.sty}{\usepackage{upquote}}{}
% use microtype if available
\IfFileExists{microtype.sty}{%
\usepackage[]{microtype}
\UseMicrotypeSet[protrusion]{basicmath} % disable protrusion for tt fonts
}{}
\IfFileExists{parskip.sty}{%
\usepackage{parskip}
}{% else
\setlength{\parindent}{0pt}
\setlength{\parskip}{6pt plus 2pt minus 1pt}
}
\usepackage{hyperref}
\hypersetup{
            pdfborder={0 0 0},
            breaklinks=true}
\urlstyle{same}  % don't use monospace font for urls
\usepackage{color}
\usepackage{fancyvrb}
\newcommand{\VerbBar}{|}
\newcommand{\VERB}{\Verb[commandchars=\\\{\}]}
\DefineVerbatimEnvironment{Highlighting}{Verbatim}{commandchars=\\\{\}}
% Add ',fontsize=\small' for more characters per line
\newenvironment{Shaded}{}{}
\newcommand{\AlertTok}[1]{\textcolor[rgb]{1.00,0.00,0.00}{\textbf{#1}}}
\newcommand{\AnnotationTok}[1]{\textcolor[rgb]{0.38,0.63,0.69}{\textbf{\textit{#1}}}}
\newcommand{\AttributeTok}[1]{\textcolor[rgb]{0.49,0.56,0.16}{#1}}
\newcommand{\BaseNTok}[1]{\textcolor[rgb]{0.25,0.63,0.44}{#1}}
\newcommand{\BuiltInTok}[1]{#1}
\newcommand{\CharTok}[1]{\textcolor[rgb]{0.25,0.44,0.63}{#1}}
\newcommand{\CommentTok}[1]{\textcolor[rgb]{0.38,0.63,0.69}{\textit{#1}}}
\newcommand{\CommentVarTok}[1]{\textcolor[rgb]{0.38,0.63,0.69}{\textbf{\textit{#1}}}}
\newcommand{\ConstantTok}[1]{\textcolor[rgb]{0.53,0.00,0.00}{#1}}
\newcommand{\ControlFlowTok}[1]{\textcolor[rgb]{0.00,0.44,0.13}{\textbf{#1}}}
\newcommand{\DataTypeTok}[1]{\textcolor[rgb]{0.56,0.13,0.00}{#1}}
\newcommand{\DecValTok}[1]{\textcolor[rgb]{0.25,0.63,0.44}{#1}}
\newcommand{\DocumentationTok}[1]{\textcolor[rgb]{0.73,0.13,0.13}{\textit{#1}}}
\newcommand{\ErrorTok}[1]{\textcolor[rgb]{1.00,0.00,0.00}{\textbf{#1}}}
\newcommand{\ExtensionTok}[1]{#1}
\newcommand{\FloatTok}[1]{\textcolor[rgb]{0.25,0.63,0.44}{#1}}
\newcommand{\FunctionTok}[1]{\textcolor[rgb]{0.02,0.16,0.49}{#1}}
\newcommand{\ImportTok}[1]{#1}
\newcommand{\InformationTok}[1]{\textcolor[rgb]{0.38,0.63,0.69}{\textbf{\textit{#1}}}}
\newcommand{\KeywordTok}[1]{\textcolor[rgb]{0.00,0.44,0.13}{\textbf{#1}}}
\newcommand{\NormalTok}[1]{#1}
\newcommand{\OperatorTok}[1]{\textcolor[rgb]{0.40,0.40,0.40}{#1}}
\newcommand{\OtherTok}[1]{\textcolor[rgb]{0.00,0.44,0.13}{#1}}
\newcommand{\PreprocessorTok}[1]{\textcolor[rgb]{0.74,0.48,0.00}{#1}}
\newcommand{\RegionMarkerTok}[1]{#1}
\newcommand{\SpecialCharTok}[1]{\textcolor[rgb]{0.25,0.44,0.63}{#1}}
\newcommand{\SpecialStringTok}[1]{\textcolor[rgb]{0.73,0.40,0.53}{#1}}
\newcommand{\StringTok}[1]{\textcolor[rgb]{0.25,0.44,0.63}{#1}}
\newcommand{\VariableTok}[1]{\textcolor[rgb]{0.10,0.09,0.49}{#1}}
\newcommand{\VerbatimStringTok}[1]{\textcolor[rgb]{0.25,0.44,0.63}{#1}}
\newcommand{\WarningTok}[1]{\textcolor[rgb]{0.38,0.63,0.69}{\textbf{\textit{#1}}}}
\setlength{\emergencystretch}{3em}  % prevent overfull lines
\providecommand{\tightlist}{%
  \setlength{\itemsep}{0pt}\setlength{\parskip}{0pt}}
\setcounter{secnumdepth}{0}
% Redefines (sub)paragraphs to behave more like sections
\ifx\paragraph\undefined\else
\let\oldparagraph\paragraph
\renewcommand{\paragraph}[1]{\oldparagraph{#1}\mbox{}}
\fi
\ifx\subparagraph\undefined\else
\let\oldsubparagraph\subparagraph
\renewcommand{\subparagraph}[1]{\oldsubparagraph{#1}\mbox{}}
\fi

% set default figure placement to htbp
\makeatletter
\def\fps@figure{htbp}
\makeatother


\date{}

\begin{document}

\hypertarget{working-on-the-problem}{%
\subsection{Working on the Problem}\label{working-on-the-problem}}

\hypertarget{measuring-time}{%
\subsubsection{Measuring Time}\label{measuring-time}}

\hypertarget{problem}{%
\paragraph{Problem}\label{problem}}

Marathon matches typically have relatively large time limits for
program's running time - between 10 and 30 seconds per test case. The
problems are computationally hard and impossible to solve optimally even
within these generous time limits, so it is often a good idea to design
your solution to use up all available runtime trying to improve its
return value.

Your program gets no time warnings from the server, so it has to manage
its timing itself. To do this, you have to be able to measure the time
elapsed accurately and in a way which is as close to server's way of
time measurement as possible. In this recipe we show how this can be
done and discuss some popular pitfalls that come along the way.

\hypertarget{solution}{%
\paragraph{Solution}\label{solution}}

To measure elapsed time, you have to use one of the standard functions
of your language that return current time. Call it once before running
your algorithm, and a second time when you need to know the elapsed
time. Take the difference between measurements to determine the running
time of the algorithm (or its part in case of several measurements).

Following are the code snippets which show the usage of time measuring
functions in languages allowed at TopCoder. Each of them defines a
function getTime() that returns current time in seconds.

C++

\begin{Shaded}
\begin{Highlighting}[]

\PreprocessorTok{#include }\ImportTok{<sys/time.h>}

\DataTypeTok{double}\NormalTok{ getTime() \{}
\NormalTok{   timeval tv;}
\NormalTok{   gettimeofday(&tv, }\DecValTok{0}\NormalTok{);}
   \ControlFlowTok{return}\NormalTok{ tv.tv_sec + tv.tv_usec * }\FloatTok{1e-6}\NormalTok{;}
\NormalTok{\}}
\end{Highlighting}
\end{Shaded}

Java

\begin{Shaded}
\begin{Highlighting}[]

\DataTypeTok{double} \FunctionTok{getTime}\NormalTok{() \{}
   \KeywordTok{return} \FloatTok{0.001}\NormalTok{*}\BuiltInTok{System}\NormalTok{.}\FunctionTok{currentTimeMillis}\NormalTok{();}
\NormalTok{\}}
\end{Highlighting}
\end{Shaded}

C\#

\begin{Shaded}
\begin{Highlighting}[]

\DataTypeTok{double} \FunctionTok{getTime}\NormalTok{() \{}
   \KeywordTok{return} \FloatTok{0.001}\NormalTok{*DateTime.}\FunctionTok{Now}\NormalTok{.}\FunctionTok{TotalMilliseconds}\NormalTok{;}
\NormalTok{\}}
\end{Highlighting}
\end{Shaded}

VB.NET

\begin{verbatim}

Function getTime As Double
   Return 0.001*Date.Now.TotalMilliseconds
End Function
\end{verbatim}

\begin{Shaded}
\begin{Highlighting}[]

\end{Highlighting}
\end{Shaded}

\hypertarget{discussion}{%
\paragraph{Discussion}\label{discussion}}

There are some general advices to be taken into account if your solution
relies heavily on time measurement.

\begin{itemize}
\item
  Remember that testing system measures total time elapsed while your
  code was running (so-called wall clock time), not CPU time used by it.
  Functions like clock() in C++ or time.clock() in Python ignore the
  time used by other processes, so using them will result in exceeding
  time limit.
\item
  Submissions are tested in sandboxed environment, so all system calls
  (including the calls of time measuring functions) are inspected by the
  testing system. This makes them slow - much slower than on your system
  - so calling them too often can slow down your solution significantly
  and rob the algorithmic part of the precious time. Thus, for example,
  if your solution does a lot of small optimizations in a loop (for
  example, in hill climbing or simulated annealing algorithms), you
  might consider calling getTime() not each iteration but every 10th or
  100th iteration (depending on how long one iteration takes).
\item
  All time measurement functions have a certain accuracy which differs
  from their precision. Thus, System.currentTimeMillis() in Java gives
  accuracy of about 10 milliseconds. Don't rely too much on time
  measurement being precise, though; remember that running your solution
  on TopCoder server has overhead you can't measure, like instantiating
  the class and passing parameters to the method. In recent matches the
  time spent on reading your return is not calculated towards the total
  runtime of your solution, but some of the overhead is still there. My
  personal preference is to play safe by leaving 0.5s of unmeasured time
  for such things. After all, changes which are done in last 20\% or
  less of the time are usually minor, and can't cover the harm of
  possibly hitting time limit and losing the score for that test case
  totally.
\end{itemize}

Local testing of solutions which rely heavily on time measurement has
some extra pitfalls in it. First, TopCoder servers measure wall clock
time, but they are focused on running the solution alone. Your computer
most likely has other things running simultaneously with your solution -
background music, instant messengers, open browser and IDE etc. This
means you'd better measure CPU time used by your process alone.

Another pitfall could be the operating system you're using, if it
differs from the one used by TopCoder. The servers which test C++, Java
and Python solutions use Linux, while the servers which test C\# and
VB.NET use Windows. If you're using Windows for competing in C++, for
example, gettimeofday() won't work for you locally. C++ allows to handle
this in a way which will work both on TC servers and on your system
without modifying the code: use \#define and \#ifdef compiler directives
to distinguish between local run and TC testing system.

\begin{Shaded}
\begin{Highlighting}[]
\PreprocessorTok{#ifdef LOCAL}

\PreprocessorTok{#include }\ImportTok{<time.h>}

\PreprocessorTok{#else}

\PreprocessorTok{#include }\ImportTok{<sys/time.h>}

\PreprocessorTok{#endif}
 

\DataTypeTok{double}\NormalTok{ getTime() \{}

\PreprocessorTok{#ifdef LOCAL}

   \ControlFlowTok{return}\NormalTok{ (clock()/CLK_TCK);            }\CommentTok{//CPU time on Windows}

\PreprocessorTok{#else}

\NormalTok{   timeval t;}
\NormalTok{   gettimeofday(&t,NULL);}
   \ControlFlowTok{return}\NormalTok{ t.tv_sec + t.tv_usec * }\FloatTok{1e-6}\NormalTok{;  }\CommentTok{//wall clock time on Linux}

\PreprocessorTok{#endif}

\NormalTok{\}}
\end{Highlighting}
\end{Shaded}

Of course, there are other time measuring routines you might prefer.
Thus, C\# has Stopwatch class:

\begin{Shaded}
\begin{Highlighting}[]
\KeywordTok{using}\NormalTok{ System.}\FunctionTok{Diagnostics}\NormalTok{;}

\NormalTok{Stopwatch sw = }\KeywordTok{new} \FunctionTok{Stopwatch}\NormalTok{();}

\NormalTok{sw.}\FunctionTok{Start}\NormalTok{();                 }\CommentTok{//start countdown}

\CommentTok{//...}

\NormalTok{time = sw.}\FunctionTok{ElapsedMilliseconds}\NormalTok{;  }\CommentTok{//get the time elapsed since the call of Start()}
\end{Highlighting}
\end{Shaded}

In Java you can use System.nanoTime() which gives nanoseconds precision
(but not accuracy!), though I don't think anybody would need that kind
of precision in a Marathon match.

In C++ you can use inline ASM to measure time via RDTSC (read time-stamp
counter) instruction, but it's a low-level utility and the results might
vary depending on the processor frequency of your computer vs TC system.
You'll have to do some kind of scaling if you want to use it:

\begin{Shaded}
\begin{Highlighting}[]
\DataTypeTok{double}\NormalTok{ getTime() \{}

   \DataTypeTok{unsigned} \DataTypeTok{long} \DataTypeTok{long}\NormalTok{ time;}

   \ExtensionTok{__asm__} \AttributeTok{volatile}\NormalTok{ (}\StringTok{"rdtsc"}\NormalTok{ : }\StringTok{"=A"}\NormalTok{ (time));}

\PreprocessorTok{#ifdef LOCAL}

   \ControlFlowTok{return}\NormalTok{ time / LOCAL_SCALE;}

\PreprocessorTok{#else}

   \ControlFlowTok{return}\NormalTok{ time / TC_SCALE;}

\PreprocessorTok{#endif}

\NormalTok{\}}
\end{Highlighting}
\end{Shaded}

\hypertarget{running-the-visualizer-with-your-solution}{%
\subsubsection{Running the Visualizer with Your
Solution}\label{running-the-visualizer-with-your-solution}}

\hypertarget{problem-1}{%
\paragraph{Problem}\label{problem-1}}

You have coded a solution for a Marathon problem, and you want to test
it locally using the provided tester before submitting it to the server.

\hypertarget{solution-1}{%
\paragraph{Solution}\label{solution-1}}

Typically each Marathon problem provides an offline tester - a tool
which allows to generate test cases, run your solution on them and score
the results without submitting it to TopCoder servers. If the problem
allows, the tester provides a visualization of the input data and the
return of your solution - an image or a dynamic drawing, sometimes
interactive; that's why the tester is usually referred to as the
visualizer. Using the tester you can gain more insight in the problem
and get more information about the performance of your solution.

The solution you submit to the server contains only the implementation
of the required class. To run it with the visualizer, you have to add
the code which will interact with the visualizer, namely read the
parameters generated by the visualizer from standard input, create an
instance of the class, pass the parameters to the corresponding method
of the class, receive the results and write them to standard output.

In some problems the scheme is more complicated: the required methods of
the class are called iteratively until a certain condition is satisfied.
Since this condition is usually checked at visualizer side, it's enough
for your code to read the parameters from stdin and write the results of
processing them to stdout in an infinite loop. Once the visualizer has
finished receiving data from your solution, it will halt the solution.

Some problems provide library methods available on the server, and using
them locally requires additional code which imitates calling these
methods and receiving their return via standard output and standard
input of your solution.

The exact order of actions is described in pseudocode in the visualizer
manual, but you have to implement them in language of your choice.

\textbf{Discussion}

To run the visualizer with your solution, you should usually run:

java -jar \emph{jarname} -exec \emph{command} -seed \emph{seed}

Here \emph{jarname} is the name of provided .jar file which contains the
visualizer, seed is seed for test case generation, and \emph{command} is
the command which runs your solution.

If your solution is written in Java and stored in file YourSol.java, you
have to compile it to YourSol.class, and use ``java YourSol'' as
\emph{command}. For other languages, compile your solution into an
executable file and use its name as \emph{command}.

Running the tester on a certain seed will always generate the same test
case, so setting it is a convenient way to check the efficiency of
different approaches on the same data. The only exception to this is
seed 0, which generates a new test case each time. Test cases generated
with seeds 1 through 10 are typically used as example tests (this will
be given in the problem statement), so you can use them to compare the
results of local testing and testing on the servers. For massive local
testing you can choose any set of seeds and stick to it whenever you
test new solution.

Some visualizers provide additional options which tune the way the
visualization is done, but these options depend on the specific problem
and are listed in the visualizer manual. Usually it is possible to turn
visualization off; for some game-based problems the visualizer might
allow to play the game yourself, i.e., provide the data for it via
visual interface.

There are a few things which should be stressed before and apart from
showing them in the examples:

\begin{itemize}
\item
  flush streams each time you have finished writing data to them,
  otherwise the visualizer doesn't start reading from them and becomes
  not responsive.
\item
  when you read data from input stream, use a method which reads the
  end-of-line character from the stream, otherwise you might get some
  weird values for next variables.
\item
  don't leave any unread data in standard input of your solution, even
  if you don't need it at the moment, otherwise the visualizer might be
  unable to read the contents of its standard output.
\item
  generally, if you're not yet comfortable with using the visualizer,
  perform the data exchange between it and your solution exactly as
  described in the manual.
\end{itemize}

There are three basic ways of visualizer-solution interaction.

\begin{enumerate}
\def\labelenumi{\arabic{enumi}.}
\tightlist
\item
  The visualizer calls the solution's method once.
\end{enumerate}

In this case your interaction code has to read the parameters from
stdin, process them, print the result to stdout and flush stdout.
Example of such problem is
\href{http://www.topcoder.com/longcontest/?module=ViewProblemStatement\&rd=14209\&pm=10756}{BrokenClayTile}.
The pseudocode given in the visualizer description for this problem is
typical and easy to follow:

\begin{verbatim}
   S = int(readLine())

   N = int(readLine())

   P = int(readLine())

   for (i=0; i<P; i++)

       pieces[i] = readLine()

   ret = reconstruct(S, N, pieces)

   for (i=0; i<S; i++)

       printLine(ret[i])

   flush(stdout)
\end{verbatim}

\begin{enumerate}
\def\labelenumi{\arabic{enumi}.}
\setcounter{enumi}{1}
\tightlist
\item
  The visualizer calls the solution's method (or several methods)
  several times.
\end{enumerate}

This is typical for game-based or simulation-based problems, in which
the task is done in several steps, and information necessary for taking
a decision about the next step is known only after the previous step is
completed. The number of calls is usually unknown beforehand and is
limited either with a constant or with some condition which must be
fulfilled in order to stop the simulation. Thus, for example, in problem
\href{http://www.topcoder.com/longcontest/?module=ViewProblemStatement\&rd=14195\&pm=10728}{ChessPuzzle}
method click is called for as long as there are valid moves left, at
most K\emph{R}C times.

Since it's the visualizer who decides to stop simulation, for your
interaction code it's enough to run an infinite loop which reads the
parameters, processes them and prints the result. Once the simulation is
over, the visualizer will halt your running solution.

Usually this approach is combined with a single call to another method
which gives initial parameters of simulation or parameters that stay the
same during all simulation (in ChesssPuzzle it's start). This initial
call should be handled as described in previous paragraph. If one of
several methods is called depending on the outcome of the completed
step, the visualizer will usually let know which one is called (via
standard input as well), so your loop will have to recognize this and
call the correct method.

The pseudocode for problem
\href{http://www.topcoder.com/longcontest/?module=ViewProblemStatement\&rd=14195\&pm=10728}{ChessPuzzle}
is as follows:

\begin{verbatim}
   K = int(readLine())

   R = int(readLine())

   for (i=0; i<R; i++)

       board[i] = readLine()

   printLine(start(K, board))

   flush(stdout)

   while (true)

       revealed = readLine()

       printLine(click(revealed))

       flush(stdout)
\end{verbatim}

\begin{enumerate}
\def\labelenumi{\arabic{enumi}.}
\setcounter{enumi}{2}
\tightlist
\item
  The visualizer calls the solution's method once and provides a library
  method which solution should call to get necessary information.
\end{enumerate}

This interaction method is the trickiest to implement in the visualizer.
It is used only if the other two methods are too unnatural for the
problem. In this case, once the visualizer has provided the initial
parameters, it waits for the solution's return. If it is a predefined
constant (usually ``?''), the visualizer interprets the next portion of
data as the parameters of the library method to be called; otherwise it
assumes that this is the final return of the solution. This is kind of a
reverse of the second way - it makes the solution call the visualizer's
method, instead of having the visualizer call solution's method.

Now let's have a closer look at how to write interaction code, using
problem
\href{http://www.topcoder.com/longcontest/?module=ViewProblemStatement\&rd=13766\&pm=10322}{ReliefMap}
as an example. It provides a library method Relief.measure(x,y) and thus
uses third way of visualizer-solution interaction. The interaction
pseudocode for the problem looks like this:

\begin{verbatim}
   double measure(x, y)

   {   printLine('?')

       printLine(x)

       printLine(y)

       flush(stdout)

       return double(readLine())

   }

   main

   {   H = int(readLine())

       for (i=0; i<H; i++)

           contourMap[i] = readLine()

       ret = getMap(contourMap)

       printLine('!')

       for (i=0; i<W*H; i++)

           printLine(ret[i])

       flush(stdout)

   }
\end{verbatim}

Here are the codes which allow the solution in any language be tested
with the visualizer. ``The solution itself'' is the solution which can
be submitted to TopCoder's server with little or no modification (this
is convenient, since such modifications are a source of errors). Two
other parts perform the data exchange with the visualizer. ``Imitation
of provided library class'' implements a simulation of the library class
with a method your solution will call; this method writes the call with
its parameters to standard out and reads the visualizer's response from
standard in. ``Main interaction code'' is code which calls the
solution's method and returns its result to the visualizer.

C++

\begin{Shaded}
\begin{Highlighting}[]

\CommentTok{/* ----- imitation of provided library class ----- */}

\PreprocessorTok{#include }\ImportTok{<iostream>}

\KeywordTok{using} \KeywordTok{namespace}\NormalTok{ std;}

\KeywordTok{class}\NormalTok{ Relief \{}

   \KeywordTok{public}\NormalTok{:}

   \AttributeTok{static} \DataTypeTok{double}\NormalTok{ measure(}\DataTypeTok{int}\NormalTok{ x, }\DataTypeTok{int}\NormalTok{ y) \{}

\NormalTok{       cout<<}\StringTok{"?"}\NormalTok{<<endl<<x<<endl<<y<<endl;}

\NormalTok{       cout.flush();}

       \DataTypeTok{double}\NormalTok{ ret;}

\NormalTok{       cin>>ret;}

       \ControlFlowTok{return}\NormalTok{ ret;}

\NormalTok{   \}}

\NormalTok{\};}

\CommentTok{/* ----- the solution itself ----- */}

\PreprocessorTok{#include }\ImportTok{<vector>}

\PreprocessorTok{#include }\ImportTok{<string>}

\KeywordTok{using} \KeywordTok{namespace}\NormalTok{ std;}

\KeywordTok{class}\NormalTok{ ReliefMap \{}

   \KeywordTok{public}\NormalTok{:}

\NormalTok{   vector<}\DataTypeTok{double}\NormalTok{> getMap(vector<string> contourMap) \{}

       \CommentTok{/* your code here */}

\NormalTok{   \}}

\NormalTok{\};}

\CommentTok{/* ----- main interaction code ----- */}

\DataTypeTok{int}\NormalTok{ main(}\DataTypeTok{int}\NormalTok{ argc, }\DataTypeTok{char}\NormalTok{* argv[])}

\NormalTok{\{   vector<string> contourMap;}

   \DataTypeTok{int}\NormalTok{ H;}

\NormalTok{   cin>>H;}

   \ControlFlowTok{for}\NormalTok{ (}\DataTypeTok{int}\NormalTok{ i=}\DecValTok{0}\NormalTok{; i<H; i++)}

\NormalTok{   \{   string t;}

\NormalTok{       cin>>t;}

\NormalTok{       contourMap.push_back(t);}

\NormalTok{   \}}

\NormalTok{   ReliefMap rm;}

\NormalTok{   vector<}\DataTypeTok{double}\NormalTok{> ret = rm.getMap(contourMap);}

\NormalTok{   cout<<}\StringTok{"!"}\NormalTok{<<endl;}

   \ControlFlowTok{for}\NormalTok{ (}\DataTypeTok{int}\NormalTok{ i=}\DecValTok{0}\NormalTok{; i<ret.size(); i++)}

\NormalTok{       cout<<ret[i]<<endl;}

\NormalTok{   cout.flush();}

   \ControlFlowTok{return} \DecValTok{0}\NormalTok{;}

\NormalTok{\}}
\end{Highlighting}
\end{Shaded}

Java

\begin{Shaded}
\begin{Highlighting}[]

\CommentTok{/* ----- imitation of provided library class ----- */}

\KeywordTok{import}\ImportTok{ java.io.*;}

\KeywordTok{class}\NormalTok{ Relief \{}

   \KeywordTok{public} \DataTypeTok{static} \DataTypeTok{double} \FunctionTok{measure}\NormalTok{(}\DataTypeTok{int}\NormalTok{ x, }\DataTypeTok{int}\NormalTok{ y) \{}

     \KeywordTok{try}\NormalTok{ \{}

       \BuiltInTok{System}\NormalTok{.}\FunctionTok{out}\NormalTok{.}\FunctionTok{println}\NormalTok{(}\StringTok{"?}\SpecialCharTok{\textbackslash{}n}\StringTok{"}\NormalTok{+x+}\StringTok{"}\SpecialCharTok{\textbackslash{}n}\StringTok{"}\NormalTok{+y);}

       \BuiltInTok{System}\NormalTok{.}\FunctionTok{out}\NormalTok{.}\FunctionTok{flush}\NormalTok{();}

       \BuiltInTok{BufferedReader}\NormalTok{ br = }\KeywordTok{new} \BuiltInTok{BufferedReader}\NormalTok{(}\KeywordTok{new} \BuiltInTok{InputStreamReader}\NormalTok{(}\BuiltInTok{System}\NormalTok{.}\FunctionTok{in}\NormalTok{));}

       \KeywordTok{return} \BuiltInTok{Double}\NormalTok{.}\FunctionTok{parseDouble}\NormalTok{(br.}\FunctionTok{readLine}\NormalTok{());}

\NormalTok{     \}}

     \KeywordTok{catch}\NormalTok{ (}\BuiltInTok{Exception}\NormalTok{ e) \{\}}

\NormalTok{   \}}

\NormalTok{\}}

\CommentTok{/* ----- the solution itself ----- */}

\KeywordTok{public} \KeywordTok{class}\NormalTok{ ReliefMap \{}

   \KeywordTok{public} \DataTypeTok{double}\NormalTok{[] }\FunctionTok{getMap}\NormalTok{(}\BuiltInTok{String}\NormalTok{[] contourMap) \{}

       \CommentTok{/* your code here */}

\NormalTok{   \}}

\CommentTok{/* ----- main interaction code - can be added to the solution class ----- */}

   \KeywordTok{public} \DataTypeTok{static} \DataTypeTok{void} \FunctionTok{main}\NormalTok{(}\BuiltInTok{String}\NormalTok{[] args) \{}

     \KeywordTok{try}\NormalTok{ \{}

       \BuiltInTok{BufferedReader}\NormalTok{ br = }\KeywordTok{new} \BuiltInTok{BufferedReader}\NormalTok{(}\KeywordTok{new} \BuiltInTok{InputStreamReader}\NormalTok{(}\BuiltInTok{System}\NormalTok{.}\FunctionTok{in}\NormalTok{));}

       \DataTypeTok{int}\NormalTok{ H = }\BuiltInTok{Integer}\NormalTok{.}\FunctionTok{parseInt}\NormalTok{(br.}\FunctionTok{readLine}\NormalTok{());}

       \BuiltInTok{String}\NormalTok{[] contourMap = }\KeywordTok{new} \BuiltInTok{String}\NormalTok{[H];}

       \KeywordTok{for}\NormalTok{ (}\DataTypeTok{int}\NormalTok{ i=}\DecValTok{0}\NormalTok{; i<H; i++)}

\NormalTok{           contourMap[i] = br.}\FunctionTok{readLine}\NormalTok{();}

\NormalTok{       ReliefMap rm = }\KeywordTok{new} \FunctionTok{ReliefMap}\NormalTok{();}

       \DataTypeTok{double}\NormalTok{[] ret = rm.}\FunctionTok{getMap}\NormalTok{(contourMap);}

       \BuiltInTok{System}\NormalTok{.}\FunctionTok{out}\NormalTok{.}\FunctionTok{println}\NormalTok{(}\StringTok{"!"}\NormalTok{);}

       \KeywordTok{for}\NormalTok{ (}\DataTypeTok{int}\NormalTok{ i=}\DecValTok{0}\NormalTok{; i<ret.}\FunctionTok{length}\NormalTok{; i++)}

           \BuiltInTok{System}\NormalTok{.}\FunctionTok{out}\NormalTok{.}\FunctionTok{println}\NormalTok{(ret[i]);}

       \BuiltInTok{System}\NormalTok{.}\FunctionTok{out}\NormalTok{.}\FunctionTok{flush}\NormalTok{();}

\NormalTok{     \}}

     \KeywordTok{catch}\NormalTok{ (}\BuiltInTok{Exception}\NormalTok{ e) \{\}}

\NormalTok{   \}}

\NormalTok{\}}
\end{Highlighting}
\end{Shaded}

C\#

\begin{Shaded}
\begin{Highlighting}[]

\CommentTok{// ----- imitation of provided library class -----}

\KeywordTok{using}\NormalTok{ System;}

\KeywordTok{public} \KeywordTok{class}\NormalTok{ Relief \{}

    \KeywordTok{public} \KeywordTok{static} \DataTypeTok{double} \FunctionTok{measure}\NormalTok{(}\DataTypeTok{int}\NormalTok{ x, }\DataTypeTok{int}\NormalTok{ y)}

\NormalTok{    \{   Console.}\FunctionTok{WriteLine}\NormalTok{(}\StringTok{"?\{2\}\{0\}\{2\}\{1\}"}\NormalTok{, x, y, Environment.}\FunctionTok{NewLine}\NormalTok{);}

\NormalTok{        Console.}\FunctionTok{Out}\NormalTok{.}\FunctionTok{Flush}\NormalTok{();}

        \KeywordTok{return} \DataTypeTok{double}\NormalTok{.}\FunctionTok{Parse}\NormalTok{(Console.}\FunctionTok{ReadLine}\NormalTok{());}

\NormalTok{    \}}

\NormalTok{\}}

\CommentTok{// ----- the solution itself -----}

\KeywordTok{public} \KeywordTok{class}\NormalTok{ ReliefMap \{}

    \KeywordTok{public} \DataTypeTok{double}\NormalTok{[] }\FunctionTok{getMap}\NormalTok{(}\DataTypeTok{string}\NormalTok{[] contourMap) \{}

        \CommentTok{// your code here}

\NormalTok{    \}}

\CommentTok{// ----- main interaction code - can be added to the solution class -----}

    \KeywordTok{private} \KeywordTok{static} \DataTypeTok{void} \FunctionTok{Main}\NormalTok{()}

\NormalTok{    \{   }\DataTypeTok{int}\NormalTok{ lineCount = }\DataTypeTok{int}\NormalTok{.}\FunctionTok{Parse}\NormalTok{(Console.}\FunctionTok{ReadLine}\NormalTok{());}

        \DataTypeTok{string}\NormalTok{[] contourMap = }\KeywordTok{new} \DataTypeTok{string}\NormalTok{[lineCount];}

        \KeywordTok{for}\NormalTok{(}\DataTypeTok{int}\NormalTok{ i = }\DecValTok{0}\NormalTok{; i < lineCount; i++)}

\NormalTok{            contourMap[i] = Console.}\FunctionTok{ReadLine}\NormalTok{();}

        \DataTypeTok{double}\NormalTok{[] result = }\KeywordTok{new} \FunctionTok{ReliefMap}\NormalTok{().}\FunctionTok{getMap}\NormalTok{(contourMap);}

\NormalTok{        Console.}\FunctionTok{WriteLine}\NormalTok{(}\StringTok{"!"}\NormalTok{);}

        \KeywordTok{foreach}\NormalTok{(}\DataTypeTok{double}\NormalTok{ d }\KeywordTok{in}\NormalTok{ result)}

\NormalTok{            Console.}\FunctionTok{WriteLine}\NormalTok{(d);}

\NormalTok{        Console.}\FunctionTok{Out}\NormalTok{.}\FunctionTok{Flush}\NormalTok{();}

\NormalTok{    \}}

\NormalTok{\}}
\end{Highlighting}
\end{Shaded}

Python

\begin{Shaded}
\begin{Highlighting}[]

\CommentTok{# ----- imitation of provided library class -----}

\ImportTok{import}\NormalTok{ sys}

\KeywordTok{class}\NormalTok{ Relief:}

   \AttributeTok{@staticmethod}

   \KeywordTok{def}\NormalTok{ measure(x, y):}

       \BuiltInTok{print} \StringTok{'?'}

       \BuiltInTok{print}\NormalTok{ x}

       \BuiltInTok{print}\NormalTok{ y}

\NormalTok{       sys.stdout.flush()}

       \ControlFlowTok{return} \BuiltInTok{float}\NormalTok{(sys.stdin.readline())    }

\CommentTok{# ----- the solution itself -----}

\KeywordTok{class}\NormalTok{ ReliefMap:}

   \KeywordTok{def}\NormalTok{ getMap(}\VariableTok{self}\NormalTok{, contourMap):}

       \CommentTok{# your code here}

       \ControlFlowTok{pass}

\CommentTok{# ----- main interaction code -----}

\NormalTok{H }\OperatorTok{=} \BuiltInTok{int}\NormalTok{(sys.stdin.readline())}

\NormalTok{contourMap }\OperatorTok{=}\NormalTok{ []}

\ControlFlowTok{for}\NormalTok{ i }\KeywordTok{in} \BuiltInTok{range}\NormalTok{(}\DecValTok{0}\NormalTok{, H):}

\NormalTok{   contourMap.append(sys.stdin.readline()[:}\OperatorTok{-}\DecValTok{1}\NormalTok{])}

\NormalTok{rm }\OperatorTok{=}\NormalTok{ ReliefMap()}

\NormalTok{ret }\OperatorTok{=}\NormalTok{ rm.getMap(contourMap)}

\BuiltInTok{print} \StringTok{'!'}

\ControlFlowTok{for}\NormalTok{ line }\KeywordTok{in}\NormalTok{ ret:}

   \BuiltInTok{print}\NormalTok{ line}

\NormalTok{   sys.stdout.flush()}
\end{Highlighting}
\end{Shaded}

VB.NET

\begin{verbatim}

Module Module1

   ' ----- imitation of provided library class -----

   Public Class Relief

       Public Shared Function measure(ByVal x As Integer, ByVal y As Integer) As Double

           Try

               Console.WriteLine("?")

               Console.WriteLine(x.ToString())

               Console.WriteLine(y.ToString())

               Console.Out.Flush()

               Dim ret As Double

               ret = Val(Console.ReadLine())

               Return ret

           Catch

               Return 0

           End Try

       End Function

   End Class

   ' ----- the solution itself -----

   Public Class ReliefMap

       Public Function getMap(ByVal contourMap As String()) As Double()

           ' your code here

       End Function

   End Class

   ' ----- main interaction code -----

   Sub Main()

       Dim H As Integer

       Try

           H = Val(Console.ReadLine())

       Catch

       End Try

       Dim contourMap(H) As String

       Try

           For i As Integer = 0 To H - 1

               contourMap(i) = Console.ReadLine()

           Next i

       Catch

       End Try

       Dim rm As ReliefMap

       rm = New ReliefMap

       Dim ret As Double()

       ret = rm.getMap(contourMap)

       For i As Integer = 0 To ret.Length() - 1

           Console.WriteLine(ret(i))

       Next i

       Console.Out.Flush()

   End Sub

End Module
\end{verbatim}

Finally, note that even if your solution correctly interacts with the
visualizer, the results of testing your solution locally and on the
server might still differ:

\begin{itemize}
\item
  check for visualizer updates which can happen when the match has
  already started. Sometimes the first version of the visualizer might
  contain some bugs, which will be found and fixed in later versions.
\item
  the visualizer doesn't implement the check for time limit violation.
  So a solution which finishes successfully locally can get ``Time Limit
  Exceeded'' when tested on the server.
\item
  on the other hand, if your solution uses certain iterative technique
  and checks its running time itself, the results can differ because on
  different systems different number of iterations is done.
\item
  floating point calculations can cause minor differences in your
  solution's behaviour on different systems, and in problems like
  \href{http://apps.topcoder.com/forums/?module=Thread\&threadID=670892\&start=0}{BounceOff}
  this might cause major differences in results.
\end{itemize}

** Rewriting the Visualizer

\hypertarget{problem-2}{%
\paragraph{Problem}\label{problem-2}}

Nowadays most Marathon problems come with visualizers, which simplify
local testing. But sometimes the official visualizer lacks some
functionality you need or is simply uncomfortable to use. In these cases
you can modify the visualizer or even write your own. This recipe will
address several issues you have to keep in mind when doing this.

\hypertarget{solution-2}{%
\paragraph{Solution}\label{solution-2}}

If you're using Java, modifying the visualizer is simple: you just take
the official one and fix whatever you need, or move pieces of code into
new one. Java is extremely convenient because you can keep the scheme of
parameters generation provided by the visualizer, which is important for
being able to reproduce the exact test case that will be generated for a
specific test case.

For other languages, you'll have to take the official one as the base
and translate it. Depending on the problem, this can be quite simple
(for example, if only math without any language-specific classes is
involved). However, you won't be able to generate exactly the same test
case (at least not in a simple way), since random numbers generators
differ for different languages. The official visualizer typically uses
the same SecureRandom class with SHA1PRNG algorithm, the same as server
tester does. To have exactly the same test cases generated, you have two
basic options:

\begin{itemize}
\item
  write your own port of this algorithm. It is possible, but quite
  time-consuming, so it's probably a thing you want to do between
  matches and not during one.
\item
  use the official visualizer once to generate input data for all test
  cases you will be using, save it to a file (or a set of files) and use
  your visualizer to load data from them, test your solution on this
  data and process the results. This doesn't include much extra work -
  after all, you've probably written the code which reads the parameters
  directly from the visualizer, uses them to drive your solution and
  passes the results back (as described in ``Running Visualizer with
  Your Solution'' recipe). You'll just have to modify this code to read
  parameters from file instead of standard input stream, or simply
  redirect input stream to be taken from file. To capture input data
  generated by the visualizer, you can modify it to write it to a file,
  or write a dummy solution which will read it, dump it to a file and
  return something that visualizer will accept as a valid return.
\end{itemize}

Note that you can use the second approach directly only for problems
which generate all parameters immediately, before first call of the
solution, for example,
\href{http://www.topcoder.com/longcontest/?module=ViewProblemStatement\&rd=10932\&pm=8426}{Permute},
\href{http://www.topcoder.com/longcontest/?module=ViewProblemStatement\&rd=14272\&pm=10942}{Planarity},
\href{http://www.topcoder.com/longcontest/?module=ViewProblemStatement\&rd=14273\&pm=10989}{CellularAutomaton}
etc. If the problem is a multi-step one, with some parameters being
generated on the go, like in
\href{http://www.topcoder.com/longcontest/?module=ViewProblemStatement\&rd=14354\&pm=10034}{StreetSales}
or
\href{http://www.topcoder.com/longcontest/?module=ViewProblemStatement\&rd=13795\&pm=10410}{TilesMatching},
you might not be able to pre-generate and capture them easily. However,
most problems which look like that, in reality pre-generate some state
randomly and then obtain parameters to be passed to the solution from
this state in a deterministic way; see, for example,
\href{http://www.topcoder.com/longcontest/?module=ViewProblemStatement\&rd=14196\&pm=10728}{ChessPuzzle},
\href{http://www.topcoder.com/longcontest/?module=ViewProblemStatement\&rd=14208\&pm=10807}{BlackBox}
or
\href{http://www.topcoder.com/longcontest/?module=ViewProblemStatement\&rd=13766\&pm=10322}{ReliefMap}.
For them, you can modify the visualizer to output this hidden state, and
build your visualizer over it.

The most important part of developing an alternate visualizer is
ensuring that the results it produces are the same as the ones produced
by official visualizer - otherwise you might spend time on improving
completely wrong thing. If your visualizer generates parameters from the
scratch, remember to compare them to the ones generated by official
visualizer; and in any case spend some time testing the scores given by
your visualizer with ones given by official one - they have to be
identical when obtained for the same solution.

\hypertarget{discussion-1}{%
\paragraph{Discussion}\label{discussion-1}}

Why could someone want to spend time on modifying the visualizer or
inventing it from scratch? There can be quite a lot of reasons:

\begin{itemize}
\item
  visualizer efficiency. The official visualizer usually copies the
  logic of the server code which tests your solutions. It is not
  optimized in any way, so in some cases it might be less efficient than
  you want it to be. For example, in
  \href{http://www.topcoder.com/longcontest/?module=ViewProblemStatement\&rd=13768\&pm=10372}{BounceOff}
  the simulation of ball movement could have been optimized, and a lot
  of people did it to win some execution time.
\item
  interaction efficiency. The official visualizer interacts with the
  solution via standard input/output, because it needs to be compatible
  with solutions in all languages allowed in the competition. In some
  matches the data passed in this way is quite large, so passing
  parameters for a single test case can take several minutes. For
  example, in
  \href{http://www.topcoder.com/longcontest/?module=ViewProblemStatement\&rd=14002\&pm=10680}{SpaceMedkit}
  2 of 5 input parameters were huge string arrays (300 KB and 43 MB
  respectively), and they didn't vary from run to run. Rewriting the
  visualizer so that it passed only varying parameters, and constant
  parameters were taken by the solution directly from file, and doing a
  batch of test cases in row (with one file read across all test cases)
  decreased time of local testing run a lot.
\item
  need for more information. The things you might want to know about
  processing of your solution can be anything - from hidden parameters
  of the test case to visualization of some extra elements of the
  problem. This is the reason why learning enough Java to be able to
  understand visualizer source code and slip a few minor modifications
  into it is a must.
\item
  tracing and debugging possibilities. In some cases it is useful to be
  able to do step-by-step tracing of either your code or visualizer's
  code on a particular test case. You can't do this using standard
  visualizer, but once you have rewritten it so that the visualizer
  calls the solution directly, you're free to use any debugging tools
  your IDE provides.
\end{itemize}

Even if your visualizer is correct, you may not get the exact same
results when comparing the official visualizer to your new one. Floating
point rounding issues could creep in, and in some problems this can
cause very different results. In these cases, creative intervention is
required to verify the operation of your runner. If you suspect this and
are using C++, try compiling both your visualizer and solution without
optimizations. This avoids register optimization passes from the
compiler which can cause some rounding errors.

\hypertarget{implementing-the-limitations-locally}{%
\subsubsection{Implementing the Limitations
Locally}\label{implementing-the-limitations-locally}}

\hypertarget{comparing-your-solutions}{%
\subsubsection{Comparing Your
Solutions**}\label{comparing-your-solutions}}

\hypertarget{problem-3}{%
\paragraph{Problem}\label{problem-3}}

You have two or more different solutions, and neither of them is the
best on all test cases. You want to choose the best of them - either for
your final submission or for further improvement.

\hypertarget{solution-3}{%
\paragraph{Solution}\label{solution-3}}

The most evident solution is just to make a full submission for each of
your solutions on TopCoder server. This will give you an idea of how
your solutions perform against other competitors' solutions, and
certainly you can simply choose the highest-scoring of your submissions.
However, this way you'll get no information on performance on individual
tests, and nothing about how your solutions score compared to each
other. Making example submissions gives detailed information on each
test case, but there are only a few tests available - definitely not
enough to make a decision about one solution being really better than
the other. To get decent information for comparison, you'll have to test
your solutions locally.

Use the visualizer or the tester you've written yourself to run each
solution on a massive batch of tests - at least a hundred, the more the
better. Keep track of the individual scores of each solution, and
calculate their scores against each other using the same method of
calculating the overall score as the one used by TopCoder server (the
one described in problem statement).

\hypertarget{discussion-2}{%
\paragraph{Discussion}\label{discussion-2}}

When you estimate the results of local testing, individual scores can be
transformed into overall solution's score in multiple ways. It's
important to use the real scoring formula for this, since other ones
might distort the results significantly. For example, if the problem
statement defines the overall score as ``the sum of individual scores,
divided by 100'', you can use the sum of raw scores without loss of
score quality. But if the problem uses any form of relative scoring, the
things are different, and you would be misled by checking just the sum
of raw scores. For example, if each test case's contribution to the
overall score is its individual score, divided by the maximal score
anybody achieved on this test case, then it's not the absolute value of
the individual score that matters, but the relative value within a test
case.

Note that the suggested method of comparing solutions is not a panacea.
Often you'll find out that all your solutions do a bad job on a
particular class of tests, which have been solved much better by someone
else. If the problem uses relative scoring, these cases' contributions
to overall scores will differ from your local estimates a lot. A good
practice is to compare locally the scores of as many your solutions as
possible - including the most trivial ones, like returning empty array
or a constant.

It is possible to gather information from doing full submission for
several different solutions and compare their scores from TopCoder
tester with locally predicted ones. Note, though, that the interval
between full submissions might distort this information, if the holder
of top scores on some test cases resubmits between two submissions of
yours.

Your scoring program should track as many information about your
solutions as possible: the quantity and seeds of failed test cases, the
best and the worst seeds etc. This is easy to do and will save you a lot
of time when tracking down why the solution fails, why it scored lower
than another one against your expectations, and what you should do to
improve it.

\hypertarget{deciding-how-to-improve-your-solution}{%
\subsubsection{Deciding How to Improve Your
Solution}\label{deciding-how-to-improve-your-solution}}

\hypertarget{problem-4}{%
\paragraph{Problem}\label{problem-4}}

There comes a time in each match where you need to decide which aspects
of your solution need improving. This may be a particular set of test
cases that you need to improve on, or a certain component within your
approach that needs refinement, and you may or may not have a good idea
exactly how much improvement is needed.

\hypertarget{solution-4}{%
\paragraph{Solution}\label{solution-4}}

While there are no hard and fast rules for how to improve your solution,
there are a few tricks that are often applicable:

\begin{itemize}
\item
  Get an approximation of the best possible score for each case.
  Sometimes it's possible to get an upper (or lower if the score must be
  minimized) bound with a few heuristics and to use this as a gold
  standard against which to compare your solutions. This can help you to
  determine which test cases you have the most to gain on, and exactly
  how much potential there is for gain on each of them.
\item
  Write a ``cheating'' solution and run it locally. In many problems
  there are hidden parameters that the tester is aware of, but your
  solution is not (typically some parameters of test case generation
  process). If your solution guesses these parameters and relies on them
  in further calculations, try to modify your tester to pass these
  parameters directly to your solution and modify your solution to use
  them instead of your guess. This can help you decide whether or not
  it's worth the extra effort to improve your parameter estimation
  process.
\item
  Extend the time limit. This is particularly helpful in optimization
  problems (for example, based on hill climbing or simulated annealing
  techniques), where you are testing a large number of potential returns
  against a scoring function. If you're trying to decide how much a 10\%
  speed improvement can help, first allow your submission to use 10\%
  more time and see whether things improve enough to make it worthwhile.
  Letting a solution run long is also another great way to get estimates
  of the best scores possible for a test case.
\item
  Separate analysis of each component of your solution. There are a lot
  of Marathon problems where your solution naturally breaks into a few
  different components, perhaps stages in a multi-stage approach.
  Putting some effort into determining which pieces of your solution are
  the weakest can have huge benefits. The analysis can range from simply
  thinking about the different pieces and how important each seems (or
  where you've been neglecting potential improvements) all the way to
  devising specific quality tests for each piece in isolation.
\item
  Watch the visualizer. You have to be careful with this one, because
  the visualizer can be as much of a distraction as it can be a useful
  tool, but very often the best way to gain insight into your solution's
  weaknesses is watch it in action. Pay careful attention to the
  decisions it makes, and how you would correct those decisions if you
  could intervene, and then see if you can modify the code to intervene
  for you.
\end{itemize}

\hypertarget{discussion-3}{%
\paragraph{Discussion}\label{discussion-3}}

It's important to remember that Marathon Matches are as much about
optimizing the time you spend on the problem as they are about
optimizing your solution. A little bit of extra time spent carefully
considering where your effort is best applied can make the time spent
applying that effort incredibly more useful. With time and practice
you'll also find that you're able to skip some of the more analytical
steps and rely a bit more on your intuition to guide you, but whenever
you find that improvements are hard to come by, take a step back and
make sure that you're working on the right things.

Now let's have a look at the problems in which the listed tricks can be
helpful.

\begin{itemize}
\tightlist
\item
  The best possible score often is evident from the problem statement,
  especially for problems which require achieving some result as closely
  as possible. For example, in
  \href{http://www.topcoder.com/longcontest/?module=ViewProblemStatement\&rd=14412\&pm=11123}{DigitsPattern}
  and
  \href{http://www.topcoder.com/longcontest/?module=ViewProblemStatement\&rd=14339\&pm=10989}{CellularAutomaton}
  you are required to get a pattern as close to the given one as
  possible; in
  \href{http://www.topcoder.com/longcontest/?module=ViewProblemStatement\&rd=14195\&pm=10728}{ChessPuzzle}
  you have to remove as many tiles as possible, ideally all of them,
  etc.
\end{itemize}

However, even in problems of this kind the evident best score sometimes
can't be reached. In
\href{http://www.topcoder.com/longcontest/?module=ViewProblemStatement\&rd=14300\&pm=10942}{Planarity},
you need to get as few pairs of intersecting edges as possible - but for
non-planar graphs you won't be able to get rid of all intersections, so
the optimal score will be strictly greater than 0. To make the estimate
more realistic depending on the graph given, one could use crossing
number inequality, which states that for any graph with V vertices and E
edges the number of edge intersections of this graph is \textgreater{}=
E-3\emph{V. Given that in this problem E was chosen between 2}V and
5*V-1, this estimate was useful for approximately two thirds of test
cases.

In problems in which the best possible score is not evident right away,
it still can be roughly estimated sometimes. In
\href{http://www.topcoder.com/longcontest/?module=ViewProblemStatement\&rd=14264\&pm=10918}{PolymerPacking}
the score is the number of segments in the polymer, divided by the area
occupied by it. The maximal score is not given explicitly, but can be
estimated (very roughly) as 2 in a following way: given an unlimited
number of mirror operations, a chain without 0 elements can be turned
into (1,1, -1,-1, 1,1, -1,-1, \ldots{}), in which each new pair of
segments adds one unit of area, making area roughly equal to L/2.

\begin{itemize}
\item
  Cases of problems in which hidden parameters need to be estimated as
  part of the solution and not as the final answer are not so frequent,
  but still exist. Thus, in
  \href{http://www.topcoder.com/longcontest/?module=ViewProblemStatement\&rd=10722\&pm=7548}{DensityImaging}
  test cases were generated using so-called ``blur factor'' - the number
  of times the blurring process was applied during test case generation.
  A lot of good solutions estimated it and used for final adjustment of
  the return. A ``cheating'' tester could pass it directly to the
  solution, and you could either check that the estimate was correct or
  watch the effect of getting it wrong.
\item
  Making the solution run faster is almost always a good idea - simply
  because it can't hurt in any way except for taking extra
  implementation time. 10\% more time won't make a critical improvement
  if your solution consists of one round of hill climbing, since most
  likely the result will have established by then. But if you restart
  your hill climbing several times and take the best result, or use any
  other approach which needs several batches of action, a 10\%
  improvement might give you time to run another batch and thus affect
  the result significantly.
\item
  Almost all good solutions consist of several stages, and more than
  once after the match people realize that they've spent days on pruning
  one stage, when they should have spent an hour on another one. The
  only thing which can be recommended here is to do a pause now and then
  and to have a fresh look at the solution.
\item
  Note that watching the visualizer can be misleading. For example, in
  \href{https://community.topcoder.com/longcontest/?module=ViewProblemStatement\&rd=14195\&pm=10728}{ChessPuzzle}
  the intuitive approach is to remove the central pieces first and the
  border and corner ones later, since there are several types of tiles
  which lead only to the border of the board. However, the top solutions
  used calculations instead of human intuition, and their moves seem
  completely counter-intuitive to the watcher until later in the game.
\end{itemize}

\hypertarget{improving-your-solution-in-the-home-stretch}{%
\subsubsection{Improving Your Solution in the Home
Stretch**}\label{improving-your-solution-in-the-home-stretch}}

\hypertarget{problem-5}{%
\paragraph{Problem}\label{problem-5}}

You are in the ``home stretch'' (usually starting Sunday evening for
those who work typical day jobs), so you cannot invest time in large
refactorings, different algorithms, or things which might not pay off,
but still want to make some improvements.

\hypertarget{solution-5}{%
\paragraph{Solution}\label{solution-5}}

Use one or several of the tried and true methods for squeezing a few
extra points out of a solution:

\begin{enumerate}
\def\labelenumi{\arabic{enumi}.}
\item
  Loop until you run out of time repeating your algorithm if it uses a
  pseudo-random generator, and halt when nearing the time limit,
  returning the best solution found so far.
\item
  Attempt solutions which use multiple algorithms (already coded ones)
  and return the best result.
\item
  Splice multiple solutions together based on local testing data and
  input parameters.
\item
  Write a script to search for optimal values of any ``tweakable''
  parameters.
\end{enumerate}

\hypertarget{discussion-4}{%
\paragraph{Discussion}\label{discussion-4}}

It's late Sunday evening and, after peering at your screen through
bleary eyes, taking the trip to the bathroom muttering strange things to
yourself and provoking stares from your children or cats, you've just
discovered The Idea. It requires a scrap and rewrite, but it will
completely dominate the scoreboard. You'll be published after this!
``This will rule them all,'' you cackle rubbing your hands together in
that sleep-deprived, maniacal way.

Well, probably not. It's important to make a reasonable estimate of how
much time you have left and how you can use it. At this stage in the
game you have one or two evenings left, and one social event, desperate
grocery stop, or call from your child's principal could easily cut that
in half or eliminate it. Often those last-minute ideas just can't pay
off (of course, there's always that one that might\ldots{})

So what can you do?

\emph{Loop until you run out of time}

If you aren't using all available time and your algorithm uses a
pseudo-random generator or can be easily adapted to use one in such a
way that it creates solutions approximately as good, on average, as it
used to, but with some deviation, then this is an easy tactic. Simply
repeat your algorithm with different seeds and record the best score (as
well as the return value which produced it).

This is applicable for problems with full information, in which you can
predict the score of a return value before actually returning it. More
specific examples include problems solvable with hill climbing, in which
you pick the initial state and/or the change of the current state
randomly. In general case running the hill climbing for 20 seconds gives
not much improvement over running it for 5 seconds, and restarting it
four times with different initial states increases the chances of
finding another local maximum, with better value.

\emph{Run multiple algorithms in series}

If your primary algorithm doesn't dominate the other algorithms you've
scored locally and you have time to run more than one, this might be an
alternative to looping until you run out of time. Run each algorithm and
return the result which produces the best score. This also applies to
problems with full information, since you need to calculate the score of
the return as well.

\emph{Splice together multiple solutions}

If you can't run multiple solutions in series, but you have a second
algorithm which scores better on some subset of cases that you can
detect with some accuracy from the input, include both algorithms and
decide which one to use based on the input. To do this, you will need to
analyze your local testing data.

Usually it is possible to write a quick brute force algorithm to solve
small test cases, so you can use different algorithms depending on the
size of the input. Alternatively, the problem might contain some
non-size parameter which affects the nature of the test case itself. For
example, in
\href{http://www.topcoder.com/longcontest/?module=ViewProblemStatement\&rd=11131\&pm=8632}{Epidemic}
there were three possible values of K - the variable which defined the
lag between people getting infected and you getting information about
this. It was very natural to process values of K = 0 (immediate gain of
knowledge) and K = 2 (two days' delay) using different algorithms - in
the first case you'd inoculate friends of the infected people, while in
the second one you'll have to figure out how far had the infection
spread already.

\emph{Find optimal values for ``tweakable'' parameters}

A lot of problems require you to simplify them by plucking somewhat
arbitrary numbers out of the air - such as the grid size for a packing
problem, or how many elements to process at once, or a probabilistic
fudge factor to account for anticipated improvements by some sort of
later optimization pass.

If you've set up your local tester, you can write a script to inject
these values into your solution and run your test set. In C++, you can
do this with the preprocessor, passing a ``-D'' option to the compiler
to set the value for each run. In Java, you can import a constant from
another class locally, and generate and compile that class with a little
bit of shell. You'll also need a script for automated estimate of
results - you don't want to choose your values by looking through
thousands of individual scores.

Since we are talking about a last minute hack that you are probably
going to kick off before you hit the sack on Sunday or Monday evening
(way later than you should have, of course), heavy strategies like
coding a genetic algorithm for all the different parameters in shell is
probably not as good an idea as it sounds (and it does sound good,
doesn't it?). You'll probably only have time to write either a binary
search for one parameter with no exit condition, or some kind of naive
hill climbing. Well, just trying a lot of values would work as well, but
that's too boring to be put here.

After you kick it off, make sure that your SSH server is up and running
and you can get to your tester box - you know, running on that odd port
so that you can get through the firewall hole that MIS probably knows
about but doesn't have time to fix. You're going to want to check on it
once in the morning to make sure it didn't go completely haywire, and
again at lunch just in case you can start the search for the next
parameter.

\end{document}
