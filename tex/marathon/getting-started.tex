\hypertarget{getting-started}{%
\subsection{Getting Started}\label{getting-started}}

This chapter introduces the basics of Marathon Matches.

\hypertarget{understanding-the-marathon-competition-format}{%
\subsubsection{Understanding the Marathon Competition
Format}\label{understanding-the-marathon-competition-format}}

\hypertarget{problem}{%
\paragraph{Problem}\label{problem}}

Individual matches in Marathon Competitions are called Marathon Matches
(abbreviated to MMs), or simply Marathons. Prior to competing in one,
one has to understand its structure and requirements.

\hypertarget{solution}{%
\paragraph{Solution}\label{solution}}

TopCoder Marathon Competition has a specific format which is somewhat
similar to other long-term on-line contests but still requires some
insight.

Each MM presents the competitors only one problem. Unlike SRM problems,
MM problems can not be solved exactly -- either because there is no
exact solution as such, or it is considered impossible to find within
given time constraints. The task of each MM is to find a solution as
close to the optimal one as possible; the quality of the solution is
estimated using a certain (problem-specific) scoring system which is
described as a part of problem statement. Score depends only on how good
your solution performs and never on how many time has passed between
reading the statement and submitting the solution.

From technical point of view, MM consists of two phases -- Submission
and System Testing. Submission phase is the time (a rather long one --
usually two weeks, sometimes one or four weeks) to solve and submit the
given problem. You can register for the match at any time during
Submission phase; the problem statement is available even to
unregistered members, so you can decide whether you will take part in a
particular match after studying the problem.

After you read the problem statement, figure out a decent way to solve
it and code the solution (all done off-line), it's time to submit your
solution. You can do two types of submits: Example Test and Full
Submission.

Example Test runs your code on a small (usually 10) predefined set of
test cases and provides you detailed information on the results: the raw
score (the individual score you've received on each test case, without
aggregation in overall score), the runtime, error messages if the test
case crashed and debug output if your solution generates it. This is
somewhat similar to testing your SRM solution on examples before
submitting it, except that you can't choose the test cases to run. You
can do Example Test at most once per 15 minutes.

Full Submission (also referred to as provisional testing) runs your code
on a larger (usually 100) predefined set of test cases and provides you
only the overall score of this submission, without per-test-case
breakout or score. You can do Full Submission at most once per 2 hours.

Current rank-list of the contest is built based on last Full Submissions
of all participants. A person is considered to be a participant of the
match not if they have registered for it but if they have done Example
Test or Full Submission. For some relative scoring systems overall
scores of all participants are recalculated after each Full Submission.
Example Tests don't affect standings; if you have done only Example
Tests in the match, for the purposes of ratings calculations you are
considered to be ranked lower than the people who scored zero on Full
submissions.

After the Submission phase ends, System Testing (or final testing)
starts. In this phase the latest Full Submissions of each competitor are
tested against even larger set of test cases, and scores on these cases
form final standings, used as match results, for rating calculations
etc.

\hypertarget{discussion}{%
\paragraph{Discussion}\label{discussion}}

Marathon Matches use a format common for long-term contests on hard
problems. A few contests similar to typical Marathon tasks are:
\href{http://azspcs.com}{Al Zimmermann's Programming Contests} and
\href{http://aichallenge.org}{Google AI Challenge}, and occasional
data-mining contests are similar to special Marathon events.

Marathon Matches don't have a tight schedule like SRMs, so they needn't
a refined strategy -- basically it's just ``solve and submit'' at your
own pace. However, there are some subtle things which should be stressed
to avoid common but painful mistakes.

Example Test isn't an actual submit -- if you do it but don't do Full
Submission, you won't get a final score, so you'll be ranked even lower
than people with zero score, with corresponding rating decrease.

Remember about the interval between Full Submissions, and plan last day
of competition accordingly. Thus, it is strongly recommended not to
postpone your last submission till last hours -- a small bug which
causes it to crash might leave you without time to resubmit and poor
total score. You should leave a few hours for a desperate resubmission,
as well as run Example Test before Full Submission -- since the
intervals between Example Tests and Full Submissions are independent,
you'll save a lot of time by checking the validity of your new
submission before fully submitting it.

The time management of the match depends heavily on schedule of your
daily life during the match. You don't have to stop all activities to do
good in a Marathon -- sometimes it's enough to get a great idea and a
few days to implement it. However, usually Marathon Matches are
associated with long-term devotion to solving single problem.

It is important to read and re-read problem statement before starting to
code, to make sure you understand what you have to do. Play with the
visualizer if you prefer visual information over textual one. Submit in
the contest only when you're 100\% sure that you'll have enough time and
inspiration to take part in the match.

To build a strategy for a particular match, you have to answer several
questions as early in the match as possible:

\begin{itemize}
\item
  How much time you're willing to spend on this problem?
\item
  What is the purpose of participation -- win, increase your rating, win
  a t-shirt (if there is one for participation or top-something
  placement), have fun playing with the problem?
\item
  Do you need to do any research before approaching the problem? If you
  do, how much time you'll spend on it?
\item
  How many different approaches you'll try?
\item
  What are fun parts of the problem? You might like the problem so much
  that you decide to take part regardless of other factors.
\item
  What is the variance of the problem and the level of randomness in the
  results? In some problems the scoring or the input data is developed
  in a way which makes final rank-list differ a lot from provisional
  one, sometimes in a pretty random way. Some people prefer to avoid
  such problems, since bad luck can spoil the performance in final
  testing even for a good solution.
\end{itemize}

\hypertarget{sidebar-understanding-specialized-marathon-matches}{%
\subsubsection{Sidebar: Understanding Specialized Marathon
Matches}\label{sidebar-understanding-specialized-marathon-matches}}

From time to time TopCoder hosts specialized Marathon Matches -- matches
which aim either to solve some real world problem or to test/promote new
hardware or software platform. Such matches usually have sponsors (the
company which needs this problem solved or platform played with) and
monetary prizes, and sometimes they require specific skills or access to
specific hardware. Note that sponsors and prizes alone are not a
criterion -- regular Marathons can have them as well, like NSA series.

Specialized matches come either as part of series or on their own. The
series which took place so far were:

\begin{itemize}
\item
  \emph{Intel Multi-Threading Competition }consisted of 12 matches which
  ran in 2006, one match per month. This series allowed usage of
  multi-threading and encouraged it, since the problems were suited for
  parallel programming. Submissions were tested on a dedicated Intel
  multi-processor server, and the only language allowed was Intel C++.
\item
  \emph{AMD Multicore Threadfest} consisted of 4 matches which ran in
  2008. This was another multi-threading series, with simulation and
  image-processing related problems. The only language allowed was C++.
\item
  \emph{CUDA Superhero Challenge} consisted of Beta contest and 2 main
  matches and ran in 2009. Yet another multi-threading series used
  nVidia CUDA API to run code on a graphics processing unit. Submissions
  were tested on a dedicated server with powerful CUDA-enabled GPU.
\item
  \emph{NASA-TopCoder} is the only ongoing series, which aims to solve
  real-life problems in an innovative way. It started in November 2009
  with
  \href{http://community.topcoder.com/longcontest/?module=ViewProblemStatement\&rd=14002\&pm=10680}{SpaceMedkit}
  problem, which required designing optimal medical kit for space
  missions. The results of the match were so impressive that in 2011
  NASA Tournament Lab was created to run matches which solve NASA tasks.
  First problem was
  \href{http://community.topcoder.com/longcontest/?module=ViewProblemStatement\&rd=14481\&pm=11313}{VehicleRecognition}
  which focused on classifying aerial photos with respect to whether
  they contained vehicles on them. The second one was
  \href{http://community.topcoder.com/longcontest/?module=ViewProblemStatement\&rd=14570\&pm=11457}{CraterDetection},
  also an image processing problem, which goal was to detect craters in
  a given set of orbital images taken under various conditions. This
  series didn't restrict the language used.
\item
  \emph{Harvard Business School} experimental series was a special one
  -- it focused not only on solving the given task but also on studying
  individual against team problem solving. The contestants were divided
  in groups, and solved the task either individually or in teams
  depending on the group type. First match of the series was
  \href{http://community.topcoder.com/longcontest/?module=ViewProblemStatement\&rd=13796\&pm=10390}{SequenceAlignment}
  in April 2009, which solved common bio-informatics task of same name.
  Second match was SpaceMedkit, shared with NASA series.
\end{itemize}

All individual specialized matches focus on real-life tasks:

\begin{itemize}
\item
  \emph{Predictive Marathon Competition 1} (June 2008) required to
  predict the outcomes of TopCoder Component competitions based on large
  set of real historical data for these competitions.
\item
  \emph{FundingPrediction} (February 2009) predicted outcome of loan
  funding based on training data set of real-life data about loans over
  a period of two years.
\item
  \emph{Linden Lab OpenJPEG} (February 2009) aimed to speed up decoding
  of ``JPEG 2000'' format in open source OpenJPEG library to use in the
  ``Second Life'' virtual world.
\item
  \emph{AgentMatching} (August 2009) was similar to FundingPrediction
  and predicted outcomes of real estate deals.
\item
  \emph{MessageDispatcher }(September 2009) simulated a real-life
  message dispatching system which processed huge quantities of messages
  in a most efficient way.
\item
  \emph{OrdinalTraitAssociationMapping }(August 2011) required to
  determine DNA markers associated with a trait, given DNA marker
  genotype data for a large number of individuals
\end{itemize}

\hypertarget{dissecting-a-problem-statement}{%
\subsubsection{Dissecting a Problem
Statement}\label{dissecting-a-problem-statement}}

\hypertarget{problem-1}{%
\paragraph{Problem}\label{problem-1}}

TopCoder Marathon problems have a specific format of statement. Being
able to grasp the essence of the problem fast is not so important as in
SRMs, since Marathons allow to spend much more time on the problem.
However, the statements tend to be larger than in SRM problems, and it's
still important to understand the structure of the statement to be more
comfortable with the problem.

\hypertarget{solution-1}{%
\paragraph{Solution}\label{solution-1}}

The problem statement consists of a set of sections, given always in the
same order. Unlike SRMs, some of the sections are optional and can be
omitted depending on the specifics of the problem and the writer's
preferences.

\emph{Statement}

The main part of the problem statement. Since it can be quite long, most
writers break it up into several smaller parts:

\begin{itemize}
\item
  Introduction is first paragraph or two give an informal introduction
  to the problem -- its background, a general idea of what it's about
  and what the task will be.
\item
  Implementation explains in detail what methods your code must
  implement in order to create a valid submission: the format and
  meaning of input parameters and return of each method, the purpose of
  each method and the order in which they will be called.
\item
  Scoring explains how the return of your solution will be evaluated --
  both how scores on individual test cases are calculated and how the
  overall score is constructed from individual scores. See recipe
  ``Understanding Absolute and Relative Scoring'' for discussion of
  scoring schemes.
\item
  Test Case Generation describes the process used to generate the test
  cases. Sometimes it describes generation of each input parameter of
  each method in detail, and sometimes it gives a high-level scheme of
  the process and refers to visualizer code for details of
  implementation. You can usually skip this section and still submit a
  valid solution, but reading it might give you an advantage of
  understanding some details which might be vital for success.
\item
  Visualizer is a standalone program that allows you to test your
  solution locally at your computer and get the same scores that
  TopCoder server would produce. Java source code of the visualizer is
  also provided, and this is extremely important because that is the
  exact and complete definition of the problem expressed in a formal
  language. If you have any doubt about the problem statement or you
  want to know minor details of the process which are omitted in the
  problem statement, the source code of the visualizer will give the
  answer. Visualizer manual is usually separated from problem statement.
\end{itemize}

Note that the division into subsections is optional -- some writers
don't use it, and some problem statements are too short for further
breaking.

\emph{Definition}

This part provides the signatures of class and methods you'll have to
submit. Problem statement is shown using your default language, which
can be set in the Arena or on the code submission page using the
website. This part, as well as the next one, is auto-generated.

\emph{Available Libraries}

Specifies the signatures of library methods you'll have to use in some
problems. These are part of interactive problems, in which you are not
just given the data to process but rather have to choose which data
you'll need -- the data you're given depends on your actions. Basically
you'll have to call library methods, the tester will generate its return
and give it to you for further processing. The methods themselves and
the context of their usage are described in Introduction.

\emph{Notes}

Things the writer wanted to mention but couldn't fit in Statement. This
part also contains notes about the testing process -- memory limit, time
limit, code size limit, numbers of test cases etc.

\emph{Constraints}

The constraints on the input parameters. This part is usually shorter
than in SRMs and give less detail, since most things about input
parameters are described in Test Case Generation section. Other than
that, they are the same as in SRMs.

\emph{Examples}

Provides the list of test cases which will be used in example testing,
usually as seeds and some core parameters of the problem. You can
generate these examples by using the visualizer with the seeds indicated
in each example. Other than that, this section is quite useless, since
the examples are not annotated and usually don't give full information
about the test cases.

\hypertarget{discussion-1}{%
\paragraph{Discussion}\label{discussion-1}}

Now let's have a closer look at how this structure works in a real
problem
\href{http://community.topcoder.com/longcontest/?module=ViewProblemStatement\&rd=14272\&pm=10942}{Planarity}.
Introduction is very short and gives the context of the problem:

You are given a graph to be drawn on the plane. All edges of the graph
are drawn as straight lines. Your task is to arrange the vertices of the
graph so that the number of intersecting pairs of edges is minimized.

This problem is simple enough, so this paragraph gives not only the
context but also the whole problem, even if in a rather informal way --
more formal specification will follow. For more complex problems, like
\href{http://community.topcoder.com/longcontest/?module=ViewProblemStatement\&rd=14354\&pm=10034}{StreetSales},
the introduction to the problem is longer and provides not only the
context but also vital information about the trading scheme used in the
problem. Generally after reading Introduction you will know what you
have to do, and the rest of the problem will provide the details on how
to do this.

Planarity is a fictional problem; sometimes problems are taken from real
life -- in such cases their real backgrounds can be given to provide
contestants extra motivation for solving them; see sidebar
``Understanding Specific Marathon Matches'' for examples.

Next part is Implementation, with descriptions of methods, their
parameters and return. Planarity has only one method to implement, so
it's rather short as well:

Your code should implement one method \textbf{untangle}(int V, vector
edges). The parameters of this method describe the graph in the
following way: \textbf{V} is the number of vertices in the graph, and
(\textbf{edges{[}2*j{]}}, \textbf{edges{[}2*j+1{]}}) are the indices of
vertices which form j-th edge. You have to return a vector which
contains the coordinates of the vertices. Elements 2\emph{i and (2}i+1)
of your return should contain the x- and y-coordinates of i-th vertex,
respectively.

Note that Implementation matches Definition exactly, it just explains
the methods in more details. Now you know how your solution must work
from tester's perspective. For problems with more methods to implement
(like StreetSales) or with library methods available (like
\href{http://community.topcoder.com/longcontest/?module=ViewProblemStatement\&rd=13766\&pm=10322}{ReliefMap})
Implementation will be longer, but it will always match Definition and
Available Libraries sections.

Next part is Test Case Generation; in Planarity it's full but relatively
short, so it doesn't refer the reader to visualizer source. In some
problems test cases are generated simply as a sequence of values sampled
from specific probability distributions (usually uniform); for examples,
see
\href{http://community.topcoder.com/longcontest/?module=ViewProblemStatement\&rd=14273\&pm=10989}{CellularAutomaton}
or
\href{http://community.topcoder.com/longcontest/?module=ViewProblemStatement\&rd=13964\&pm=10655}{EnclosingCircles}.
In most problems, however, there will be some more complex algorithm
involved; see, for example,
\href{http://community.topcoder.com/longcontest/?module=ViewProblemStatement\&rd=12203\&pm=8737}{Textures},
StreetSales, ReliefMap or nearly any other problem.

Scoring section follows; in Planarity it's trivial, but sometimes it can
be the most massive part of the problem. For example, in
CellularAutomaton all you have to do is return a new initial
configuration of the automaton, but to score your return, the automaton
evolution has to be followed for several steps. Same goes for
\href{http://community.topcoder.com/longcontest/?module=ViewProblemStatement\&rd=13791\&pm=10372}{BounceOff}
-- you return a set of obstacles to be placed, and the score is based on
the results of simulation of ball movement.

As usual in more recent problems, Planarity visualizer is moved outside
of the problem statement, to a
\href{https://www.topcoder.com/contest/problem/Planarity/manual.html}{separate
page}. It has links to visualizer .jar file and source code .java file,
a pseudocode of what your program has to do to interact with it and a
description of commands and options used to run visualizer.

In Notes you can find all kinds of things which didn't fit in the
Statement:

\begin{itemize}
\item
  problem statement clarifications:
  \href{http://community.topcoder.com/longcontest/?module=ViewProblemStatement\&rd=13679\&pm=10242}{KnightsMoveCipher},
  \href{http://community.topcoder.com/longcontest/?module=ViewProblemStatement\&rd=11131\&pm=8632}{Epidemic};
\item
  scoring of invalid return and details of test case generation (if not
  mentioned in corresponding sections):
  \href{http://community.topcoder.com/longcontest/?module=ViewProblemStatement\&rd=13709\&pm=10032}{MegaParty};
\item
  external sources used to generate test cases: KnightsMoveCipher,
  Epidemic,
  \href{http://community.topcoder.com/longcontest/?module=ViewProblemStatement\&rd=13564\&pm=9906}{OneTimePad}.
\end{itemize}

Compared to SRMs problems, Marathons statements tends to be less formal
and to rely on visualizer for details; Notes are more about organization
of the testing process than about problem clarifications, Constraints
give less details, and Examples tend to be pretty useless without the
visualizer.

A final recommendation: be sure to read the problem statement very
carefully -- it seems an obvious thing to do but this is an important
step to solving the correct problem in a correct way. Usually revisiting
the problem statement page in a later stage of the competition is a good
idea, since something that looked like a minor detail at first could be
used to improve your solution.

\hypertarget{submitting-your-solution}{%
\subsubsection{Submitting Your
Solution}\label{submitting-your-solution}}

\hypertarget{problem-2}{%
\paragraph{Problem}\label{problem-2}}

In the previous recipe we have reviewed the structure of Marathon
problem statement. Once you've figured out what the problem is and how
you are going to solve it, you have to actually code the solution and
submit it. Same as SRMs, Marathons require the submitted code to be in
specific format.

\hypertarget{solution-2}{%
\paragraph{Solution}\label{solution-2}}

You can submit for a Marathon (as well as register for the match and
check current standings) both on the website (Competitions
-\textgreater{} Marathon Match -\textgreater{} Active Contests) and in
Arena (Active Contests -\textgreater{} current match).

Marathons allow you to choose among the same four languages as SRMs --
C++, Java, C\# and VB.NET -- and add Python as an option. For some ideas
on which language to choose, see sidebar ``Choosing the Right
Programming Language''. If you're using Arena, Definition and Available
Libraries will change accordingly to your chosen language, so you will
be able to see what data types to use and how the declare the required
methods. On the website you can change your preferred language in the
submission page.

With respect to solution format, Marathon problems are somewhat similar
to SRM ones. You must submit implementation of one class, as specified
in Definition part of problem statement. Unlike SRM problems, the
implementation might require multiple public methods to function
properly. In this case, problem statement will specify the order in
which they will be called. Your code can also contain any accessory
classes, but all code must be submitted in a single file, so, for
example, in Java they can not be declared as public. Any supporting data
structures you need can be included in the file either as global
variables (if your language allows them) or as class variables. Note
that each test case is tested in a separate process using one instance
of the class, so you can use global variables and class variables for
storing data between method calls within one test, but not between
different tests.

Another difference from SRMs is that in some problems you have to use
``libraries'' provided by the problem environment -- methods which give
you extra information about the problem. Their signatures are also given
in problem statement, in section ``Available libraries''.

The return from your solution is evaluated in a different way than in
SRMs: instead of just checking it for being identical with a certain
correct answer, it is evaluated with respect to how good it solves the
given problem. In some cases this is done by comparing with ``perfect''
answer, and the score is the measure of similarity to it; however, in
most cases the quality of your return is evaluated on its own.

The solution has limitations for its execution time and memory
consumption, which vary from problem to problem and are specified in the
statement. Your code doesn't have to be readable, but some problems
impose a limitation on its size.

\hypertarget{discussion-2}{%
\paragraph{Discussion}\label{discussion-2}}

Here are skeleton submissions for
problem\href{http://community.topcoder.com/longcontest/?module=ViewProblemStatement\&rd=14273\&pm=10989}{CellularAutomaton}
in all available languages. The basic solution for this problem is very
simple -- you are required to return some initial configuration of the
automaton, and the easiest way to do this is to return the configuration
you received as parameter.

C++

\begin{Shaded}
\begin{Highlighting}[]

\PreprocessorTok{#include }\ImportTok{<string>}
\PreprocessorTok{#include }\ImportTok{<vector>}

\KeywordTok{using} \KeywordTok{namespace}\NormalTok{ std;}

\KeywordTok{class}\NormalTok{ CellularAutomaton \{}
\KeywordTok{public}\NormalTok{:}

\NormalTok{   vector<string> configure(vector<string> grid, string rules, }\DataTypeTok{int}\NormalTok{ N, }\DataTypeTok{int}\NormalTok{ K) \{}
       \ControlFlowTok{return}\NormalTok{ grid;}
\NormalTok{   \}}
\NormalTok{\};}
\end{Highlighting}
\end{Shaded}

Java

\begin{Shaded}
\begin{Highlighting}[]

\KeywordTok{public} \KeywordTok{class}\NormalTok{ CellularAutomaton \{}

   \KeywordTok{public} \BuiltInTok{String}\NormalTok{[] }\FunctionTok{configure}\NormalTok{(}\BuiltInTok{String}\NormalTok{[] grid, }\BuiltInTok{String}\NormalTok{ rules, }\DataTypeTok{int}\NormalTok{ N, }\DataTypeTok{int}\NormalTok{ K) \{}

       \KeywordTok{return}\NormalTok{ grid;}

\NormalTok{   \}}

\NormalTok{\}}
\end{Highlighting}
\end{Shaded}

C\#

\begin{Shaded}
\begin{Highlighting}[]

\KeywordTok{using}\NormalTok{ System;}

\KeywordTok{public} \KeywordTok{class}\NormalTok{ CellularAutomaton \{}

   \KeywordTok{public} \DataTypeTok{string}\NormalTok{[] }\FunctionTok{configure}\NormalTok{(}\DataTypeTok{string}\NormalTok{[] grid, }\DataTypeTok{string}\NormalTok{ rules, }\DataTypeTok{int}\NormalTok{ N, }\DataTypeTok{int}\NormalTok{ K) \{}
       \KeywordTok{return}\NormalTok{ grid;}
\NormalTok{   \}}

\NormalTok{\}}
\end{Highlighting}
\end{Shaded}

VB.NET

\begin{verbatim}

Public Class CellularAutomaton

   Public Function configure(ByVal grid As String(), ByVal rules As String, _
        ByVal N As Integer, ByVal K As Integer) As String()

       Return grid

   End Function

End Class
\end{verbatim}

Python

\begin{Shaded}
\begin{Highlighting}[]

\KeywordTok{class}\NormalTok{ CellularAutomaton:}

   \KeywordTok{def}\NormalTok{ configure(}\VariableTok{self}\NormalTok{, grid, rules, N, K):}

       \ControlFlowTok{return}\NormalTok{ grid}
\end{Highlighting}
\end{Shaded}

\hypertarget{sidebar-choosing-the-right-programming-language}{%
\subsubsection{Sidebar: Choosing The Right Programming
Language}\label{sidebar-choosing-the-right-programming-language}}

TopCoder Marathon competitions support five programming languages: C++,
C\#, Java, Visual Basic.NET and Python. What language to choose as main
language for competitions, and when it's worth to temporary switch
language for particular marathon match?

First, let's look at the stats for last 25 marathon matches (Marathon
Match 45 -- Marathon Match 72, only regular marathon matches were
considered, and matches 57, 66 and 70 were skipped because they have too
many first-placed submissions).

\begin{longtable}[]{@{}llllll@{}}
\toprule
& C++ & Java & C\# & VB.NET & Python\tabularnewline
\midrule
\endhead
1-st places & 14 & 7 & 3 & 1 & 0\tabularnewline
Top 3 & 45 & 20 & 8 & 2 & 0\tabularnewline
Top 10 & 160 & 62 & 25 & 3 & 0\tabularnewline
F1 score & 1589 & 644 & 251 & 41 & 0\tabularnewline
\bottomrule
\end{longtable}

The last row is score according to the recent
\href{https://en.wikipedia.org/wiki/List_of_Formula_One_World_Championship_points_scoring_systems}{Formula
One World Championship points scoring system}.

Of course, the results are biased towards more popular languages, but
they probably gained their popularity for a reason.

Although generally Python is more popular (has more submissions) than
VB.NET, it earns zero points. The main reason for that is that, unlike
TopCoder SRM competitions, in Marathon Matches speed of a solution does
play a huge role, and Python is the only interpreted language from five
supported languages (besides, TopCoder doesn't support
\href{http://psyco.sourceforge.net}{Psyco} -- a Python extension module
which can greatly speed up the execution of a program). So even though
Python is officially supported in Marathon competitions, it's really not
an option if you want to win or take a good place (unless your algorithm
is really superb, but there is no historical evidence of such case and
you probably would have even bigger advantage using faster language). If
you are Python expert, it might be handy for you to write a prototype
program in Python and then rewrite it in a faster language.

In the above table C++ language has more points than all other languages
added together. The main reason is speed again: C++ is considered the
fastest of all five supported languages (and usually you can use MMX,
SSE and inline assembler to speed up it even more). So in the long run
C++ is probably the language of choice if you want to win in TopCoder
Marathon Matches.

Unlike SRMs, particular features of some language like arbitrary
precision numbers or regular expression support in Java will not give a
huge advantage, because Marathon Matches are long enough to implement
needed piece of functionality in any language (although it can be handy
to have it in standard library of a language). Java has another minor
advantage -- in most Marathon Matches pieces of visualizer code
(provided as part of the problem) can be reused because it is written in
Java.

On the other hand, Marathon Matches programs are much larger and much
more complex compared to SRM solutions, and are more like real life
programming. C\# and Java are more high-level languages than C++, so
they can help to manage complexity better because of higher level of
abstraction (and TopCoder doesn't support popular Boost C++ Libraries
that extend the functionality of C++). So some problems -- the ones
which require more complex implementation and are not very
computation-heavy -- might favor using Java or C\# over C++.

As for non-standard, specific Marathon Matches, the above considerations
also apply, but choices are usually more limited. Many specific
Marathons require to use C/C++ only (Intel Multi-Threading Competition
series, AMD Multicore Threadfest, Linden Lab OpenJPEG).

To summarize::

\begin{itemize}
\item
  you can't really use Python for your final submissions if you want to
  take high place;
\item
  you can use C/C++ in every Marathon Match (including special format
  contests);
\item
  you might want to use more high level languages like Java or C\# for
  some problems that require complex implementation and are not very
  computation-heavy.
\end{itemize}

Also when considering different languages find in the rules or ask in
the forums of the particular contest about what specific versions of
compilers/interpreters server uses, in that environment and with what
options, what libraries are allowed. TopCoder system doesn't support the
latest versions of the allowed languages, as well as some popular
libraries, and you have to take this into account when making your
choice.

\hypertarget{understanding-marathon-scoring-systems}{%
\subsubsection{Understanding Marathon Scoring
Systems}\label{understanding-marathon-scoring-systems}}

\hypertarget{problem-3}{%
\paragraph{Problem}\label{problem-3}}

Marathon problems are typically impossible to solve exactly under the
given constraints. The participants' submissions are scored based on
their efficiency in solving the given problem. The scoring schemes used
to do this vary from match to match, and can sometimes be quite
complicated. Understanding them is important to be able to focus on the
best strategy for the match. This recipe explains the way scoring is
done in Marathons, and examines the common scoring schemes.

\hypertarget{solution-3}{%
\paragraph{Solution}\label{solution-3}}

For each problem there is a set of test cases on which all solutions are
tested. The performance of the solution on each test case is estimated
with a numeric value called ``individual score''. After the solution is
tested on all test cases in the set, its individual scores are
accumulated to get a measure of how good the solution is compared to
solutions of other competitors -- another numeric value called ``overall
score''.

The methods of calculating both individual and overall scores for each
problem are defined in ``Scoring'' section of the problem statement.
Individual scores are a measure of how good is the solution itself on
each test case, and the way they are calculated depends strongly on the
nature of the problem. However, there are two basic types of ways to
calculate overall scores:

\begin{enumerate}
\def\labelenumi{\arabic{enumi}.}
\item
  Absolute scoring. The contribution of a test case to overall score of
  the solution depends only on its score for this test case. A typical
  example of absolute scoring is calculating overall score of the
  submission as a sum or average of its individual scores for all test
  cases.
\item
  Relative scoring. The contribution of a test case to overall score of
  the submission depends not only on its score for this test case, but
  also on the scores the other submissions got for this test case.
\end{enumerate}

\hypertarget{discussion-3}{%
\paragraph{Discussion}\label{discussion-3}}

Absolute scoring is usually used when individual scores are already
normalized, i.e., the maximal possible score for each test case is known
beforehand.

For example, in crypto matches
\href{http://community.topcoder.com/longcontest/?module=ViewProblemStatement\&rd=12202\&pm=9798}{XORPlusEncryption},
\href{http://community.topcoder.com/longcontest/?module=ViewProblemStatement\&rd=13564\&pm=9906}{OneTimePad}
and
\href{http://community.topcoder.com/longcontest/?module=ViewProblemStatement\&rd=13679\&pm=10242}{KnightsMoveCipher}
individual score is basically the percentage of correct characters in
the decoded message, which is normalized to lie in {[}0..1{]} range
naturally. In
\href{http://community.topcoder.com/longcontest/?module=ViewProblemStatement\&rd=14209\&pm=10756}{BrokenClayTile}
the score is the percentage of tile pixels guessed correctly. In
\href{http://community.topcoder.com/longcontest/?module=ViewProblemStatement\&rd=13565\&pm=10014}{Klondike}
individual score is 1 if the game was completed successfully, and 0
otherwise.

Another case of absolute scoring is applied when the maximal possible
score for a test case is unknown beforehand, but the task is to minimize
the score. In this case the absolute score is the sum of inverses of
individual scores (possibly multiplied by a constant), like in
\href{http://community.topcoder.com/longcontest/?module=ViewProblemStatement\&rd=13797\&pm=10390}{SequenceAlignment}
and
\href{http://community.topcoder.com/longcontest/?module=ViewProblemStatement\&rd=13772\&pm=10328}{J2KDecode}.

Relative scoring is usually used when there is no simple estimate of how
large or how small the individual scores can get (see, for example,
\href{http://community.topcoder.com/longcontest/?module=ViewProblemStatement\&rd=10815\&pm=7789}{ContinuousSameGame}
or
\href{http://community.topcoder.com/longcontest/?module=ViewProblemStatement\&rd=12198\&pm=8472}{FactoryManager}).
The usual purposes of relative scoring are to make all test cases have
approximately even weights in the overall score and to prevent the
overall score from growing too large.

The most popular cases of relative scoring define the contribution of
individual test case in the overall score as either YOUR/MAX or
MIN/YOUR, where YOUR is the solution's score on this test case, and MAX
(MIN) is the maximal (minimal) score achieved by anyone on this test
case.

Other scoring schemes that are appropriate for specific contests may be
adopted, which can use more complicated math formulas which include mean
and standard deviation of top 20 scores on this test case (see
\href{http://www.topcoder.com/longcontest/?module=ViewProblemStatement\&rd=10815\&pm=7789}{ContinuousSameGame}).
The most exotic method of relative scoring so far was used in problem
\href{http://www.topcoder.com/longcontest/?module=ViewProblemStatement\&rd=10845\&pm=7892}{Navigator},
in which each test case added MIN/YOUR to overall score of submission,
but only if this submission touched the maximal number of waypoints (or
tied with some other submission to do this).

A special case of relative scoring is so-called ranking scoring, which
calculates the contribution of a test case based on the number of
submissions beaten by this one or tied with it on this test case (see
\href{https://community.topcoder.com/longcontest/?module=ViewProblemStatement\&rd=13795\&pm=10410}{TilesMatching}
or
\href{https://community.topcoder.com/longcontest/?module=ViewProblemStatement\&rd=14272\&pm=10942}{Planarity}.
It gives less information about the relative performance of your
solution against others, since there is no way to figure out by how much
on average you beat them, or how much better you have to do to catch up
with leaders -- you know just that you have to do better.

Note that 0 is usually a special score which represents some kind of
submission failure (timeout, invalid format of return, crash etc.), so
all scoring schemes disallow it to contribute towards the overall score.

With absolute scoring, it's your absolute improvement on each case that
matters, while with relative scoring it's usually your percent
improvement. Most people tend to like absolute scoring, mainly because
with it the competitor's score depends only on his submission, and not
on the other submissions, and it's much easier to track down your
overall progress.

With relative scoring, overall scores of all competitors are
recalculated after each submission. The non-evident part of this scheme
is that only the last submission of each competitor is taken into
account; maximal or minimal scores achieved by previous submissions
don't matter, and the overall scores of previous submissions (can be
seen in competitor's ``Submission history'') are not updated. This way
your overall score changes frequently (unfortunately, mostly decreases)
even if you don't submit a new solution. A good idea is either to
compare your solutions locally (see recipes ``Comparing Solutions'' and
``Keeping Track of Your Progress'') or at least save your old
submission's score before submitting the new one and compare the results
immediately.

Absolute scoring is usually cleaner and easier to use as a measure of
one's progress. In fact, the only people who really like relative
scoring are problem writers, for the main reason that it doesn't require
inventing a way to normalize individual scores.
